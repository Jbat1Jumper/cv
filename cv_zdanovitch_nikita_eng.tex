\documentclass[10pt,a4paper]{article}
\usepackage[a4paper,margin=0.75in]{geometry}
\usepackage[spanish, english]{babel}
\usepackage[utf8]{inputenc}
\hyphenation{Some-long-word}

\usepackage{resume}
\usepackage{tikz}

\newcommand{\ExternalLink}{%
    \tikz[x=1.2ex, y=1.2ex, baseline=-0.05ex]{% 
        \begin{scope}[x=1ex, y=1ex]
            \clip (-0.1,-0.1) 
                --++ (-0, 1.2) 
                --++ (0.6, 0) 
                --++ (0, -0.6) 
                --++ (0.6, 0) 
                --++ (0, -1);
            \path[draw, 
                line width = 0.5, 
                rounded corners=0.5] 
                (0,0) rectangle (1,1);
        \end{scope}
        \path[draw, line width = 0.5] (0.5, 0.5) 
            -- (1, 1);
        \path[draw, line width = 0.5] (0.6, 1) 
            -- (1, 1) -- (1, 0.6);
        }
    }


\begin{document}
\sloppy


\maintitle{Zdanovitch Nikita}{Full Stack Software Developer}{Last update at \today}

\nobreakvspace{0.3em}

\noindent
Buenos Aires \sbull
Argentina \\
Nationality: Russian \sbull
Marital status: Single \\
Phone: +54 11 2403 9260 \sbull
Mail: \href{mailto:nzdanovitch.at.gmail.dot.com}{nzdanovitch\mbox{}@\mbox{}gmail.com}

\spacedhrule{1.2em}{-0.4em}



\roottitle{Summary}

\vspace{-1.3em}
\begin{multicols}{2}
\noindent
A Computer Science student at the Faculty of Exact and Natural Sciences of the University of Buenos Aires, I also have 4 years of experience working in software development.
I am constantly looking for problems to puzzle my mind, so I can use them as an excuse to not sleep and solve them.
I am always looking for new technologies to try and explore innovative ways to develop software.
I'm interested in working professionally in the ambit of R\&D and game design, since nowadays it already is a hobby in my spare time.
\end{multicols}

\spacedhrule{0.3em}{-0.4em}



\roottitle{Knowledge}

\inlineheadsection
  {Specialized in:}
  {Programming in Python, C\# and Javascript. Software design and architecture. Unix, virtualization and continuous integration.}

\inlineheadsection
  {With experience in:}
  {Programming in C/C++, Haxe, PHP, Node, Haskell and Java. \LaTeX. Game development. Frontend development with HTML5, CSS, Less/Sass, JQuery, Ember.js and Meteor.js. Sql databases and MongoDb.}

  \vspace{0.5em}
\inlineheadsection
  {Languages:}
  {Spanish \emph{(native)}, Russian \emph{(native)}, English \emph{(intermediate)}, Portuguese \emph{(basic)}.}

\spacedhrule{1.9em}{-0.4em}



\roottitle{Academic Formation}

\vspace{0.2em}
\headedsection
  {Faculty of Exact and Natural Sciences of the University of Buenos Aires}
  {\textsc{}} {%
  \headedsubsection
    {Undergraduate Bachelor of Computer Science}
    {2013 --\ Today \ }{}
}

\vspace{0.2em}
\headedsection
  {Technical High School Nº1 'Otto Krause', Buenos Aires}
  {\textsc{}} {%
  \headedsubsection
    {Electronics technician}
    {2007 -- 2012} {}
}

\vspace{0.2em}
\headedsection
  {Courses}
  {\textsc{}} {%
  \headedsubsection
    {Protecting the data confidentiality and integrity on the web \href{http://www.dc.uba.ar/events/eci/2015/cursos/russo}{\ExternalLink}}
    {2015}{\\
    ECI, dictated in the Faculty of Exact and Natural Sciences of the UBA}
  \headedsubsection
    {Current methods on machine learning \href{http://www.dc.uba.ar/events/eci/2014/cursos/granitto}{\ExternalLink}}
    {2014}{\\
    ECI, dictated in the Faculty of Exact and Natural Sciences of the UBA}
  \headedsubsection
    {Graph-based Representation and Reasoning in Artificial Intelligence \href{http://www.dc.uba.ar/events/eci/2013/cursos/curso-croitoru}{\ExternalLink}}
    {2013}{\\
    ECI, dictated in the Faculty of Exact and Natural Sciences of the UBA}
  \headedsubsection
    {Java programming course}
    {2012}{\\
    'Digital Natives' program, dictated in the ET Nº1 'Otto Krause'}
}


\spacedhrule{0.8em}{-0.4em}

\roottitle{Working Experience}

\vspace{0.2em}
\headedsection
    {Trocafone S.A.}
    {\textsc{}}{
    \headedsubsection
        {Full Stack Developer}
        {2016 -- Today}
        {\\I'm currently working at one of the must successful e-commerces in Brazil, using a wide stack of technologies like Python, PHP, Node and Docker}
}

\vspace{0.2em}
\headedsection
    {Infobiz S.A.}
    {\textsc{}}{
    \headedsubsection
        {Software Developer}
        {2013 -- 2016}
        {\\Worked on software and games for video lottery systems, using technologies like .Net/Mono and HTML5 canvas \href{http://www.infobiz.com.ar/english/ticket_games.php}{\ExternalLink}}
}

\vspace{0.2em}
\headedsection
    {Exo S.A.}
    {\textsc{}}{
    \headedsubsection
        {Trainee Software Developer}
        {2013}
        {}
}

\vspace{0.2em}
\headedsection
    {Urban Technologies}
    {\textsc{}}{
    \headedsubsection
        {Freelance Software Developer}
        {2013 - 2014}
        {\\I developed a turn management system for queues and waiting rooms, made in Python with HTML5 frontend \href{http://www.eitsa.com.ar/producto.php?c=3&p=14}{\ExternalLink}}
}

\pagebreak

\spacedhrule{0.8em}{-0.4em}

\roottitle{Other Experiences and Projects}

\vspace{0.2em}
\headedsection
    {Nirvana Sky \href{http://globalgamejam.org/2016/games/nirvana-sky}{\ExternalLink}}
    {\textsc{}}{
    \headedsubsection
        {Mobile videogame for Global Game Jam}
        {2016}
        {\\Developed in 48 hours with Godot Engine}
}

\vspace{0.2em}
\headedsection
    {Triple Quest \href{http://globalgamejam.org/2015/games/triple-quest}{\ExternalLink}}
    {\textsc{}}{
    \headedsubsection
        {Mobile videogame for Global Game Jam}
        {2015}
        {\\Developed in 48 hours with Unity3d}
}

\vspace{0.2em}
\headedsection
    {\href{}{National Olympiad of Electronics and Telecommunications}}
    {\textsc{}}{
    \headedsubsection
        {Double Gold Medal in theoretical competition}
        {2012}
        {\\In individual and group competition, held in Blas Pascal University, Córdoba, Argentina}
}

\end{document}

＀
