\documentclass[10pt,a4paper]{article}
\usepackage[a4paper,margin=0.75in]{geometry}
\usepackage[spanish, english]{babel}
\usepackage[utf8]{inputenc}
\hyphenation{Some-long-word}
\usepackage{resume}
\usepackage{tikz}

\newcommand{\ExternalLink}{%
    \tikz[x=1.2ex, y=1.2ex, baseline=-0.05ex]{% 
        \begin{scope}[x=1ex, y=1ex]
            \clip (-0.1,-0.1) 
                --++ (-0, 1.2) 
                --++ (0.6, 0) 
                --++ (0, -0.6) 
                --++ (0.6, 0) 
                --++ (0, -1);
            \path[draw, 
                line width = 0.5, 
                rounded corners=0.5] 
                (0,0) rectangle (1,1);
        \end{scope}
        \path[draw, line width = 0.5] (0.5, 0.5) 
            -- (1, 1);
        \path[draw, line width = 0.5] (0.6, 1) 
            -- (1, 1) -- (1, 0.6);
        }
    }




\begin{document}
\sloppy


\maintitle{Zdanovitch Nikita}{Full Stack Software Developer}{Last update at \today}

\nobreakvspace{0.3em}

\noindent
Buenos Aires \sbull
Argentina \\
Nationality: Argentinian / Russian \sbull
Marital status: Single \\
Phone: +54 911 2403 9260 \sbull
Mail: \href{mailto:nzdanovitch.at.gmail.dot.com}{nzdanovitch\mbox{}@\mbox{}gmail.com} \sbull
Github: \href{http://github.com/Jbat1Jumper}{Jbat1Jumper}


\spacedhrule{1.2em}{-0.4em}  % ...............................................................................




\roottitle{Summary}

\vspace{-1.3em}
\begin{multicols}{2}
\noindent

Passionate about my work, I'm a software developer with almost 10 light-years
of mileage. My deepest and most peculiar interests seems to be always
aligned with knowledge representation, learning theory, games or all of
them at the same time.

I am constantly looking for problems to puzzle over, so I can use them as an
excuse to not sleep and learn new things. Aligned with that, I'm always
looking for new technologies to try and explore innovative ways to develop
software.



% A Computer Science student at the Faculty of Exact and Natural Sciences of
% the University of Buenos Aires. Nowadays I have 7 years of experience working
% in the area of software development.
% 
% On the one hand, I have special academic interest in the field of artificial
% intelligence, particularly KR and learning theory. On the other hand, I am
% passionate about software architecture and meta programming.
% 
% Much of my spare time goes to chimerical projects which work as fuel for
% learning and one of my goals is to dedicate myself to combine science and
% playful experiences.


%
%
% I'm interested in working professionally in the ambit of R\&D and game design, since nowadays it already is a hobby in my spare time.

\end{multicols}


\spacedhrule{0.3em}{-0.4em} % ...............................................................................



\roottitle{Academic Formation}

\vspace{0.2em}
\headedsection
  { Faculty of Exact and Natural Sciences of the University of Buenos Aires }
  { \textsc{} } {%
  \headedsubsection
    { Undergraduate Bachelor of Computer Science }
    { 2013 --\ Today \ }{}
}

\vspace{0.2em}
\headedsection
  { Technical High School Nº1 'Otto Krause', Buenos Aires }
  { \textsc{} } {%
  \headedsubsection
    { Electronics technician }
    { 2007 -- 2012 } {}
}


\vspace{0.2em}

\headedsection
  { Courses }
  { \textsc{} } {%
  \headedsubsection
    { Java programming course }
    { 2012 }{ \\ 'Digital Natives' program, dictated in the THS Nº1 'Otto Krause' }
}


\iffalse \headedsection
  { ECIs and optional assignatures }
  { \textsc{}} {%
  \headedsubsection
    { Natural language processing with nerual networks }
    { 2019 }{ \\ ECI, dictated in the Faculty of Exact and Natural Sciences of the UBA }
  \headedsubsection
    { Neural deep learning }
    { 2019 }{ \\ ECI, dictated in the Faculty of Exact and Natural Sciences of the UBA }
  \headedsubsection
    { Probabilistic classifiers in machine learning }
    { 2019 }{ \\ ECI, dictated in the Faculty of Exact and Natural Sciences of the UBA }
  \headedsubsection
    { About belief change }
    { 2019 }{ \\ Optional assignature of the Faculty of Exact and Natural Sciences of the UBA }
  \headedsubsection
    { Graphics processors architecture and applications }
    { 2018 }{ \\ ECI, dictated in the Faculty of Exact and Natural Sciences of the UBA }
  \headedsubsection
    { Computational methods for studying language in the brain }
    { 2018 }{ \\ ECI, dictated in the Faculty of Exact and Natural Sciences of the UBA }
  \headedsubsection
    { Protecting the data confidentiality and integrity on the web }
    { 2015 }{ \\ ECI, dictated in the Faculty of Exact and Natural Sciences of the UBA }
  \headedsubsection
    { Current methods on machine learning }
    { 2014 }{ \\ ECI, dictated in the Faculty of Exact and Natural Sciences of the UBA }
  \headedsubsection
    { Graph-based Representation and Reasoning in Artificial Intelligence }
    { 2013 }{ \\ ECI, dictated in the Faculty of Exact and Natural Sciences of the UBA }
} \fi


\spacedhrule{0.8em}{-0.4em} % ...............................................................................





\roottitle{Knowledge}

\inlineheadsection
  { Specialized in: }
  { Programming in C\#, Python and Rust. Software design and architecture. Unix, virtualization and continuous integration.}

\inlineheadsection
  { With experience in: }
  { Programming in C/C++, Haxe, Haskell, Java, Node, PHP, Node and Prolog. \LaTeX, OpenNLP and Protobuf. Videogame development. Frontend development with HTML5, CSS, Less/Sass, JQuery, Ember.js and Angular. Sql and Mongo databases. }

  \vspace{0.5em}
\inlineheadsection
  { Languages: }
  { Spanish \emph{(native)}, Russian \emph{(native)}, English \emph{(intermediate)} and Portuguese \emph{(basic)}.}


\spacedhrule{1.9em}{-0.4em} % ...............................................................................



\newcommand{\jobdetail}[2]{

    \small
    \leftskip=10pt
    \rightskip=60pt
    #1
    
    \vspace{5pt}
    \footnotesize
    #2

}

\roottitle{ Working Experience }

\vspace{0.2em}
\headedsection
    { Software Engineer }
    { \textit{Oct 2020 -- \ \ \ Today\ \ \ \  } }{
    \headedsubsection
        { Trocafone S.A. }
        {}
        {
            \jobdetail{
                As a software engineer at Trocafone my work consists mostly on
                developing features for the in-house ERP software.
            }{
                PHP - Laravel - Postgres - Typescript - Node.js - NestJS - Jenkins - Docker
            }
        }
}

\vspace{0.2em}
\headedsection
    { Inria - Parietal Team }
    { \textsc{} }{
    \headedsubsection
        { Research Intern }
        { Mar 2020 -- Sep 2020 }
        { 
            \jobdetail{
                As a research intern at the Parietal team my goal was to develop a domain specific language based on natural English to use as an interface to NeuroLang. Neurolang is an open-source python library that enables the analysis of neuro-imaging data through probabilistic logic programming. It seamlessly allow to combine, images, databases, and ontologies within a single framework.

                During my internship I studied how to translate sentences from a subset of first order logic into datalog and then how to interpret a controlled subset of English grammar into the former subset using discourse representation theory.

                I developed a module in Python for the NeuroLang package consisting of an implementation of a non-deterministic parser and the transformations required for a small subset of the query language. It is sill an ongoing work and part of my computer science thesis at University of Buenos Aires.
            }{
                Python - NeuroLang - Datalog - DSL - Open Source
            }

        }
}

\vspace{0.2em}
\headedsection
    { ADGS }
    { \textsc{} }{
    \headedsubsection
        { Software Developer }
        { Nov 2018 -- Feb 2020 }
        { 
            \jobdetail{
                As a software developer at ADGS I worked helping to develop an unstructured search, semantic extraction and machine learning framework called TASMO. My job consisted primarily on developing a set of criteria for entity extraction from unstructured text and for connotation analysis for medical, financial and legal texts.
            }{
                C\# - .Net Framework - Solr - OpenNLP - GATE - AWS - Azure
            }
        }
}

\vspace{0.2em}
\headedsection
    { Trocafone S.A. }
    { \textsc{} }{
    \headedsubsection
        { Full Stack Developer }
        { Jun 2016 -- Jan 2019 }
        { 
            \jobdetail{
                As a software developer at Trocafone my work consisted mostly on developing features for the in-house ERP software. My focus was given to create opportunities of technological improvement seeking better efficiency in the SCM area. Between one of the biggest projects I had the pleasure to work on was the development and deployment in production of a microservice to centralize stock management.
            }{
                PHP - Laravel - Python - Flask - Postgres - Jenkins - Docker
            }
        }
}

\vspace{0.2em}
\headedsection
    { Infobiz S.A. }
    { \textsc{} }{
    \headedsubsection
        { Software Developer }
        { Nov 2013 -- Feb 2016 }
        {
            \jobdetail{
                As a software developer at Infobiz (now Interprod) I helped with the development of the Ticket Games system  and some of the games that run on it. Ticket Games is a system similar to a slot machine but with a pre-printed outcome \href{https://web.archive.org/web/20180903134404/http://infobiz.com.ar/english/ticket_games.php}{\ExternalLink}.

                

                Some of my contributions to the project where the following:

                Developed a library to print tickets on proprietary printer hardware and interface with anti-tampering sensors
                Developed a GTK\# application for the management of fonts and images loaded to the printer
                Developed a dashboard with Ember.js for monitoring hardware status and system events in the machines
                Collaborated with the architecture design of the whole system
                Developed and maintained a CI pipeline for various projects

                Apart from the work on Ticket Games, I worked on the migration from VB to VB.NET of an accounting software solution for gas stations.
            }{
                C\# - .NET Framework - Mono - GTK\# - Ubuntu - HTML5 - Jenkins - Ember.js - VB.NET
            }
        }
}

\vspace{0.2em}
\headedsection
    { Exo S.A. }
    { \textsc{} }{
    \headedsubsection
        { Trainee Software Developer }
        { Mar 2013 -- Jun 2013 }
        {
            \jobdetail{
                As a trainee developer at EXO I had the pleasure to work with and learn from a small but very talented team helping to customize open source educational software.
            }{
                Python
            }
        }
}

\vspace{0.2em}
\headedsection
    { Urban Technologies }
    { \textsc{} }{
    \headedsubsection
        { Freelance Software Developer }
        { 2013 - 2014}
        {
            \jobdetail{
                Turn management system for queues and waiting rooms, made in Python
                and HTML5 \href{https://web.archive.org/web/20180905152241/www.eitsa.com.ar/producto.php?c=3&p=14}{\ExternalLink}.
            }{
                Python - Flask - SqlServer - Unix
            }
        }
}

\pagebreak

\spacedhrule{0.8em}{-0.4em} % ...............................................................................

\roottitle{ Other Experiences and Projects }

\vspace{0.2em}
\headedsection
    { Poncho II: Pónchosis \href{https://globalgamejam.org/2018/games/poncho-ii-p\%C3\%B3nchosis}{\ExternalLink} }
    { \textsc{} }{
    \headedsubsection
        { Mobile videogame for Global Game Jam }
        { 2018 }
        { \\ Developed in 48 hours with Unity3d }
}

\vspace{0.2em}
\headedsection
    { Nirvana Sky \href{http://globalgamejam.org/2016/games/nirvana-sky}{\ExternalLink} }
    { \textsc{} }{
    \headedsubsection
        { Mobile videogame for Global Game Jam }
        { 2016 }
        { \\ Developed in 48 hours with Godot Engine }
}

\vspace{0.2em}
\headedsection
    { Triple Quest \href{http://globalgamejam.org/2015/games/triple-quest}{\ExternalLink} }
    { \textsc{} }{
    \headedsubsection
        { Mobile videogame for Global Game Jam }
        { 2015 }
        { \\ Developed in 48 hours with Unity3d }
}

\vspace{0.2em}
\headedsection
    { \href{}{National Olympiad of Electronics and Telecommunications} }
    { \textsc{} }{
    \headedsubsection
        { Double Gold Medal in theoretical competitions }
        { 2012 }
        { \\ In individual and group categories -- held in Blas Pascal University, Córdoba, Argentina }
}


\end{document}

＀
