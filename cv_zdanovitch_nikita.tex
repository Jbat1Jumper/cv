\documentclass[10pt,a4paper]{article}
\usepackage[a4paper,margin=0.75in]{geometry}
\usepackage[english, spanish]{babel}
\usepackage[utf8]{inputenc}
\hyphenation{Some-long-word}
\usepackage{resume}
\usepackage{tikz}

\newcommand{\ExternalLink}{%
    \tikz[x=1.2ex, y=1.2ex, baseline=-0.05ex]{% 
        \begin{scope}[x=1ex, y=1ex]
            \clip (-0.1,-0.1) 
                --++ (-0, 1.2) 
                --++ (0.6, 0) 
                --++ (0, -0.6) 
                --++ (0.6, 0) 
                --++ (0, -1);
            \path[draw, 
                line width = 0.5, 
                rounded corners=0.5] 
                (0,0) rectangle (1,1);
        \end{scope}
        \path[draw, line width = 0.5] (0.5, 0.5) 
            -- (1, 1);
        \path[draw, line width = 0.5] (0.6, 1) 
            -- (1, 1) -- (1, 0.6);
        }
    }




\begin{document}
\sloppy


\maintitle{Zdanovitch Nikita}{Desarrollador de Software}{Última actualización el \today}

\nobreakvspace{0.3em}

\noindent
Domicilio: Ángel Justiniano Carranza 2139 \sbull
Capital Federal, Palermo \\
Nacionalidad: Argentino / Ruso \sbull
Estado civil: Soltero sin hijos \\
Teléfono: 15-2403-9260 \sbull
Mail: \href{mailto:nzdanovitch.at.gmail.dot.com}{nzdanovitch\mbox{}@\mbox{}gmail.com}


\spacedhrule{1.2em}{-0.4em} % ...............................................................................




\roottitle{Resumen}

\vspace{-1.3em}
\begin{multicols}{2}
\noindent

Estudiante de Licenciatura en Ciencias de la Computación de la Facultad de
Ciencias Exactas y Naturales de la UBA. Actualmente cuento con 7 años de
experiencia en el área de desarrollo de software.

Por un lado, tengo especial interés académico en el ámbito de inteligencia
artificial, en particular KR y teoría del aprendizaje. Por otro, me apasionan
la arquitectura de sistemas y la metaprogramación.

El tiempo libre lo dedico a quiméricos proyectos que funcionan como
combustible para el aprendizaje y una de mis metas es dedicarme a combinar
ciencia con experiencias lúdicas.


% Estoy constantemente buscando problemas para tener excusas para no dormir y poder resolverlos mientras tanto,
% así como también pasarme horas probando tecnologías nuevas.
% 
% Me interesa el ámbito de R\&D como también el de desarrollo de videojuegos, las cuales son cosas a las que les
% suelo dedicar mi tiempo libre y me gustaría dedicarme profesionalmente.

\end{multicols}


\spacedhrule{0.3em}{-0.4em} % ...............................................................................



\roottitle{Formación académica}

\vspace{0.2em}
\headedsection
  { Facultad de Ciencias Exactas y Naturales de la UBA }
  { \textsc{} } {%
  \headedsubsection
    { Licenciatura en Ciencias de la Computación }
    { 2013 --\ Hoy \ }{}
}

\vspace{0.2em}
\headedsection
  { Escuela Técnica N\textordmasculine1 ``Otto Krause'' }
  { \textsc{} } {%
  \headedsubsection
    { Especialidad Electrónica con orientación en Telecomunicaciones }
    { 2007 -- 2012 } {}
}


\vspace{0.2em}

\headedsection
  { Cursos }
  { \textsc{} } {%
  \headedsubsection
    { Curso de programación en Java }
    { 2012 }{ \\ Programa Nativos Digitales - Dictado en la ET N\textordmasculine1 ``Otto Krause'' }
}


\iffalse \headedsection
  { ECIs y materias optativas }
  { \textsc{} } {%
  \headedsubsection
    { Procesamiento del lenguaje natural con redes neuronales }
    { 2019 }{ \\ ECI, dictado en la Facultad de Ciencias Exactas de la UBA }
  \headedsubsection
    { Aprendizaje profundo por refuerzo }
    { 2019 }{ \\ ECI, dictado en la Facultad de Ciencias Exactas de la UBA }
  \headedsubsection
    { Clasificadores probabilisticos en aprendizaje automático }
    { 2019 }{ \\ ECI, dictado en la Facultad de Ciencias Exactas de la UBA }
  \headedsubsection
    { Sobre cambio de creencias }
    { 2019 }{ \\ Optativa de computación en la Facultad de Ciencias Exactas de la UBA }
  \headedsubsection
    { Arquitectura de procesadores gráficos y aplicaciones }
    { 2018 }{ \\ ECI, dictado en la Facultad de Ciencias Exactas de la UBA }
  \headedsubsection
    { Métodos computacionales para estudiar el lenguaje en el cerebro }
    { 2018 }{ \\ ECI, dictado en la Facultad de Ciencias Exactas de la UBA }
  \headedsubsection
    { Protegiendo la confidencialidad e integridad de datos en la web:\\ \indent Construcción segura de páginas y servidores web }
    { 2015 }{ \\ ECI, dictado en la Facultad de Ciencias Exactas de la UBA }
  \headedsubsection
    { Métodos actuales en machine learning }
    { 2014 }{ \\ ECI, dictado en la Facultad de Ciencias Exactas de la UBA }
  \headedsubsection
    { Graph-based Representation and Reasoning in Artificial Intelligence }
    { 2013 }{ \\ ECI, dictado en la Facultad de Ciencias Exactas de la UBA }
} \fi


\spacedhrule{0.8em}{-0.4em} % ...............................................................................





\roottitle{Conocimientos}

\inlineheadsection
  { Especializado en: }
  { Programación en C\#, Python y Rust. Diseño e implementación de software. Sistemas operativos linux. Virtualización e integración continua. }

\inlineheadsection
  { Con experiencia en: }
  { Programación en C/C++, Haxe, Haskell, Java, Node, PHP y Prolog. \LaTeX, OpenNLP y Protobuf. Desarrollo de videojuegos. Desarrollo de frontend web en HTML5, CSS, Less/Sass, JQuery, Ember.js y Angular. Bases de datos Sql y Mongo. }

  \vspace{0.5em}
\inlineheadsection
  { Idiomas: }
  { Español \emph{(nativo)}, Ruso \emph{(nativo)}, Ingles \emph{(intermedio)} y Portugués \emph{(básico)}. }


\spacedhrule{1.9em}{-0.4em} % ...............................................................................





\roottitle{ Experiencia laboral }

\vspace{0.2em}
\headedsection
    { Inria - Equipo Parietal }
    { \textsc{} }{
    \headedsubsection
        { Investigador Pasante }
        { 2020 -- Hoy }
        { \\ Desarrollo de un DSL basado en inglés natural para NeuroLang \href{https://neurolang.github.io/}{\ExternalLink}  }
}

\vspace{0.2em}
\headedsection
    { ADGS }
    { \textsc{} }{
    \headedsubsection
        { Desarrollador de Software }
        { 2019 -- 2020 }
        { \\ Ontologías para motor de análisis semántico en .Net \href{http://www.tasmo.ai/}{\ExternalLink} }
}

\vspace{0.2em}
\headedsection
    { Trocafone S.A. }
    { \textsc{} }{
    \headedsubsection
        { Full Stack Developer }
        { 2016 -- 2018 }
        { \\ Sistema de gestión en PHP y microservicio de manejo de stock en Python y Docker }
}

\vspace{0.2em}
\headedsection
    { Infobiz S.A. }
    { \textsc{} }{
    \headedsubsection
        { Desarrollador de Software }
        { 2013 -- 2016 }
        { \\ Sistema de juegos de video lotería utilizando tecnologías .Net/Mono y HTML5 \href{http://www.infobiz.com.ar/english/ticket_games.php}{\ExternalLink} }
}

\vspace{0.2em}
\headedsection
    { Exo S.A. }
    { \textsc{} }{
    \headedsubsection
        { Desarrollador de Software Junior }
        { 2013 }
        {}
}

\vspace{0.2em}
\headedsection
    { Urban Technologies }
    { \textsc{} }{
    \headedsubsection
        { Desarrollador Freelance }
        { 2013 - 2014 }
        { \\ Sistema de gestión de turnos para filas y salas de espera en Python e interfaz web HTML5 \href{http://www.eitsa.com.ar/producto.php?c=3&p=14}{\ExternalLink} }
}

\pagebreak

\spacedhrule{0.8em}{-0.4em} % ...............................................................................

\roottitle{ Experiencias y projectos }

\vspace{0.2em}
\headedsection
    { Poncho II: Pónchosis \href{https://globalgamejam.org/2018/games/poncho-ii-p\%C3\%B3nchosis}{\ExternalLink} }
    { \textsc{} }{
    \headedsubsection
        { Videojuego mobile para la Global Game Jam }
        { 2018 }
        { \\ Desarrollado en 48 con el tema ``Transmission''. }
}

\vspace{0.2em}
\headedsection
    { Nirvana Sky \href{http://globalgamejam.org/2016/games/nirvana-sky}{\ExternalLink} }
    { \textsc{} }{
    \headedsubsection
        { Videojuego mobile para la Global Game Jam }
        { 2016 }
        { \\ Desarrollado en 48 con el tema ``Ritual''. }
}

\vspace{0.2em}
\headedsection
    { Triple Quest \href{http://globalgamejam.org/2015/games/triple-quest}{\ExternalLink} }
    { \textsc{} }{
    \headedsubsection
        { Videojuego mobile para la Global Game Jam }
        { 2015 }
        { \\ Desarrollado en 48 con el tema ``What do we do now?'' }
}

\vspace{0.2em}
\headedsection
    { \href{}{Olimpiada Nacional de Electrónica y Telecomunicaciones} }
    { \textsc{} }{
    \headedsubsection
        { Medallas de oro en competencias teóricas }
        { 2012 }
        { \\ En las categorías grupal e individual -- Dictada en la Universidad Blas Pascal, Córdoba }
}


\end{document}

＀
