%
% LaTeX source of my resume
% =========================
%
% Heavily commented to to fit even LaTeX beginners (hopefully).
%
% See the `README.md` file for more info.
%
% This file is licensed under the CC-NC-ND Creative Commons license.
%


% Start a document with the here given default font size and paper size.
\documentclass[10pt,a4paper]{article}

% Set the page margins.
\usepackage[a4paper,margin=0.75in]{geometry}

% Setup the language.
\usepackage[spanish]{babel}
\usepackage[utf8]{inputenc}
\hyphenation{Some-long-word}

% Makes resume-specific commands available.
\usepackage{resume}




\begin{document}  % begin the content of the document
\sloppy  % this to relax whitespacing in favour of straight margins


% title on top of the document
\maintitle{Zdanovitch Nikita}{Desarrollador de Software}{Última actualización el \today}

\nobreakvspace{0.3em}  % add some page break averse vertical spacing

% \noindent prevents paragraph's first lines from indenting
% \mbox is used to obfuscate the email address
% \sbull is a spaced bullet
% \href well..
% \\ breaks the line into a new paragraph
\noindent
%\href{mailto:nzdanovitch.at.gmail.dot.com}{nzdanovitch\mbox{}@\mbox{}gmail.com}\sbull
%\textsmaller{}15-2403-9260\sbull
%{\newnums cies010} \emph{(Skype)}\sbull
%\href{http://www.abc.com/}{www.abc.com/} \\
Domicilio: Condarco 3193 \sbull Capital Federal, Villa del Parque \\
Nacionalidad: Ruso \sbull
Estado civil: Soltero sin hijos \\
Teléfono: 15-2403-9260 \sbull
Mail: \href{mailto:nzdanovitch.at.gmail.dot.com}{nzdanovitch\mbox{}@\mbox{}gmail.com}
% \thinspace {\large \sc A}\sbull


\spacedhrule{1.2em}{-0.4em}  % a horizontal line with some vertical spacing before and after


\roottitle{Resumen}  % a root section title

\vspace{-1.3em}  % some vertical spacing
\begin{multicols}{2}  % open a multicolumn environment
\noindent
%\emph{Creative geek with roots in the open source movement, an entrepreneurial mindset and a passion for delivering value by developing maintainable software.}
%\\
%\\
Estudiante de Licenciatura en Ciencias de la Computación de la Facultad de Ciencias Exactas y Naturales de la UBA.
Cuento con 3 años de experiencia en el campo de IT centrado en el desarrollo de software.
Estoy constantemente buscando problemas para tener excusas para no dormir y poder resolverlos mientras tanto,
así como también pasarme horas probando tecnologías nuevas.
Me interesa el ámbito de R\&D como también el de desarrollo de videojuegos, las cuales son cosas a las que les
suelo dedicar mi tiempo libre y me gustaría dedicarme profesionalmente.
\end{multicols}


\spacedhrule{0.3em}{-0.4em}

\roottitle{Conocimientos}

\inlineheadsection  % special section that has an inline header with a 'hanging' paragraph
  {Especializado en:}
  {Programación en C\#, Pyhton y Javascript. Diseño e implementación de software. Sistemas operativos linux. Virtualización e integración continua.}

\inlineheadsection  % special section that has an inline header with a 'hanging' paragraph
  {Con experiencia en:}
  {Programación en C/C++, Haxe, PHP, Node, Haskell y Java. \LaTeX. Desarrollo de juegos en Construct2, Unity y Godot Engine. Desarrollo de frontend web en HTML5, CSS, Less/Sass, JQuery, Ember.js, Meteor.js. Bases de datos MySql/SqlServer y MongoDb.}

  \vspace{0.5em}
\inlineheadsection
  {Idiomas:}
  {Español \emph{(nativo)}, Ruso \emph{(nativo)} e Ingles \emph{(intermedio)}.}


\spacedhrule{1.9em}{-0.4em}

\roottitle{Formación académica}

\vspace{0.2em}
\headedsection
  {Facultad de Ciencias Exactas y Naturales de la UBA}
  {\textsc{}} {%
  \headedsubsection
    {Licenciatura en Ciencias de la Computación}
    {2013 --\  Hoy \ }{}
  \headedsubsection
    {Ciclo Básico Común}
    {2012 -- 2013}{}
}

\vspace{0.2em}
\headedsection
  {Escuela Técnica N\textordmasculine1 ``Otto Krause''}
  {\textsc{}} {%
  \headedsubsection
    {Especialidad Electrónica}
    {2009 -- 2012} {}
}

\vspace{0.2em}
\headedsection
  {Escuela Técnica N\textordmasculine25 ``Fray Luis Beltran''}
  {\textsc{}} {%
  \headedsubsection
    {Ciclo Básico}
    {2006 -- 2009} {}
}

\vspace{0.2em}
\headedsection
  {Cursos}
  {\textsc{}} {%
  \headedsubsection
    {Protegiendo la confidencialidad e integridad de datos en la web:\\
    \indent Construcción segura de páginas y servidores web}
    {2015}{\\
    ECI, dictado en la Facultad de Ciencias Exactas de la UBA}
  \headedsubsection
    {Métodos actuales en machine learning}
    {2014}{\\
    ECI, dictado en la Facultad de Ciencias Exactas de la UBA}
  \headedsubsection
    {Graph-based Representation and Reasoning in Artificial Intelligence}
    {2013}{\\
    ECI, dictado en la Facultad de Ciencias Exactas de la UBA}
  \headedsubsection
    {Curso de programación en Java}
    {2012}{\\
    Programa Nativos Digitales - Dictado en la ET N\textordmasculine1 ``Otto Krause''}
}


\spacedhrule{0.8em}{-0.4em}

\roottitle{Experiencia laboral}

\vspace{0.2em}
\headedsection
    {Trocafone S.A.}
    {\textsc{}}{
    \headedsubsection
        {Full Stack Developer}
        {2016 -- Hoy}
        {\\Desarrollo de la plataforma de uno de los e-commerce más exitosos de Brasil. Usando un amplio stack cloud con tecnologías como Python, PHP, Node y Docker}
}

\vspace{0.2em}
\headedsection
    {Infobiz S.A.}
    {\textsc{}}{
    \headedsubsection
        {Desarrollador de Software}
        {2013 -- 2016}
        {\\Desarrollo de software para sistema de juegos de video lotería utilizando tecnologías .Net/Mono y HTML5}
}

\vspace{0.2em}
\headedsection
    {Exo S.A.}
    {\textsc{}}{
    \headedsubsection
        {Desarrollador de Software Junior}
        {2013}
        {}
}

\vspace{0.2em}
\headedsection
    {Urban Technologies}
    {\textsc{}}{
    \headedsubsection
        {Desarrollador Freelance}
        {2013}
        {\\Sistema de gestión de turnos para filas y salas de espera en Python e interfaz web HTML5}
}

\pagebreak

\spacedhrule{0.8em}{-0.4em}

\roottitle{Experiencias y projectos}

\vspace{0.2em}
\headedsection
    {\href{http://globalgamejam.org/2016/games/nirvana-sky}{Nirvana Sky}}
    {\textsc{}}{
    \headedsubsection
        {Videojuego mobile para la Global Game Jam}
        {Enero 2016 -- Hoy}
        {\\Desarrollado en 48 con el tema ``Ritual''. Actualmente en desarrollo.}
}

\vspace{0.2em}
\headedsection
    {\href{http://globalgamejam.org/2015/games/triple-quest}{Triple Quest}}
    {\textsc{}}{
    \headedsubsection
        {Videojuego mobile para la Global Game Jam}
        {Enero 2015}
        {\\Desarrollado en 48 con el tema ``What do we do now?''}
}

\vspace{0.2em}
\headedsection
    {\href{}{Olimpiada Nacional de Electrónica y Telecomunicaciones}}
    {\textsc{}}{
    \headedsubsection
        {Medalla de oro en competencia teórica}
        {Octubre 2012}
        {\\En las categorías grupal e individual -- Dictada en la Universidad Blas Pascal, Córdoba}
}

%\headedsection  % sets the header for the section and includes any subsections
%  {\href{http://www.hoppinger.com}{Hoppinger}}
%  {\textsc{Rotterdam, The Netherlands}} {%
%  \headedsubsection
%    {Head of Technology}
%    {Apr \apo12 -- present}
%    {\bodytext{Hoppinger is an open source minded ``full-service'' internet agency.  Reporting directly to the general director, Marijn Bom. In charge of drawing and carrying out the vision for the tech department consisting of 15 developers.  Streamlined datacenter operations with Puppet, introduced Rails for custom web-app development and Capistrano for deployment automation.  Intimately involved with the software architecture of all technically challenging projects.}}
%}
%
%\headedsection  % sets the header for a subsection and contains usually body text
%  {\href{http://www.hro.nl}{HRO} (Rotterdam University of Applied Science)}
%  {\textsc{Rotterdam, The Netherlands}} {%
%  \headedsubsection
%    {Guest Lecturer}
%    {Sep \apo12 -- present}
%    {\bodytext{Introductory lecture on history of software development and open source for 1\textsuperscript{st} year CS students.}}
%}
%
%\headedsection
%  {\href{http://www.intellecap.com}{Intellecap}/\href{http://istpl.in}{\acr{ISTPL}}}
%  {\textsc{Mumbai, Pune \& Hyderabad, India}} {%
%
%  \headedsubsection
%    {\acr{IT} Consultant}
%    {Nov \apo08 -- Feb \apo09}
%    {\bodytext{Intellecap is a social-sector advisory firm serving corporates, non-profits, development agencies and governments working in developing markets. Assessed their software development team and methodologies, trained their developers and build several web applications.  One of those apps is \href{http://www.mostfit.org}{Mostfit}, an open source \acr{MIS} for \href{http://en.wikipedia.org/wiki/Microcredit}{microcredit} lenders.}}
%
%  \headedsubsection
%    {\acr{IT} \& Strategy Consultant}
%    {Jan \apo10 -- Aug \apo11}
%    {\bodytext{Called in to solve several technical challenges and look at potential growth strategies for \href{http://www.mostfit.org}{Mostfit}.}}
%
%  \headedsubsection
%    {CTO}
%    {Oct \apo11 -- Feb \apo12}
%    {\bodytext{Proudly joined the \acr{C}-family of Intellicap's software division, ISTPL, to make \href{http://www.mostfit.org}{Mostfit} the nr.1 software solution for micro credit lenders around the globe.  Contracts got terminated half a year later due to investment issues.}}
%}
%
%\headedsection
%  {\href{http://www.zarafa.com}{Zarafa}}
%  {\textsc{Delft, The Netherlands}} {%
%  \headedsubsection
%    {\acr{QA} \& Release Manager}
%    {Dec \apo09 -- Jan \apo11}
%    {\bodytext{Zarafa might be the fastest growing open source product company in Europe, making a drop-in replacement for MS Exchange.  Reported directly to the \acr{CEO}, Brian Josef, and worked closely with the \acr{CTO}, Steve Hardy.  In charge of the 6 men strong QA department.  Established test automation and continuous integration.  Architected and implemented an all-integrated documentation and translation system that employed community effort.  Got sent to India to analyse and streamline their outsourced operations.}}
%}
%
%\headedsection
%  {\href{http://www.dharmapublishing.com}{Dharma Publishing}}
%  {\textsc{near San Francisco (\acr{CA}), \acr{USA}}} {%
%  \headedsubsection
%    {\acr{IT} Consultant}
%    {Nov \apo09 -- Dec \apo09}
%    {\bodytext{Dharma Publishing, the worlds largest Buddhist publisher, is a non-profit, all-volunteer organisation that helps to preserve Tibetan Buddhism and culture. Built their \href{http://www.dharmapublishing.com}{web shop}, and moved their digital content sales to SaaS applications.}}
%}
%
%\headedsection
%  {\href{http://www.kde.org}{KDE}}
%  {\href{http://edu.kde.org/kturtle}{edu.kde.org/kturtle}} {%
%  \headedsubsection
%    {Software Engineer}
%    {Dec \apo03 -- present}
%    {\bodytext{\acr{KT}urtle is an educational programming environment that simplifies learning the basics of programming.  \acr{KT}urtle is intended as a gift to future generations:\ a simple environment to get started with programming.  In 2003 \acr{KT}urtle got admitted to the \acr{KDE} project.}}
%}
%
%\headedsection
%  {\href{http://truetopiaproject.org}{Truetopia Project}}
%  {\href{http://truetopiaproject.org}{truetopiaproject.org}} {%
%  \headedsubsection
%    {Initiator}
%    {Nov \apo07 -- Apr \apo10}
%    {\bodytext{The Truetopia Project is an open source web application (Rails) to facilitate self-governing communities.  It provides a workflow for collaborative problem identification and solution design.}}
%}
%
%\headedsection
%  {\href{http://www.dpu.ac.th/dpuic}{Dhurakij Pundit University International College}}
%  {\textsc{Bangkok, Thailand}} {%
%  \headedsubsection
%    {Guest Lecturer}
%    {Sep \apo09}
%    {\bodytext{Invited by Dr.\@ Pilun Piyasirivej and Mr.\@ Michel Bauwens for two guest lectures:\ the open source movement and the semantic web.}}
%}
%
%\headedsection
%  {\href{http://www.opendream.th}{Opendream}}
%  {\textsc{Bangkok, Thailand}} {%
%  \headedsubsection
%    {\acr{IT} Consultant}
%    {Aug \apo09 -- Sep \apo09}
%    {\bodytext{Architected and largely implemented an open source media sharing web service (\acr{REST} api) that facilitates video uploads, transcoding and streaming.  Coached their development team on system design, Ruby development (using Merb/Rails) and testing strategies such as \acr{TDD}/\acr{BDD}.}}
%}
%
%\headedsection
%  {\href{http://www.commuun.nl}{Commuun}}
%  {\textsc{Rotterdam, The Netherlands}} {%
%  \headedsubsection
%    {Senior Visionary}
%    {Jul \apo06 -- Sep \apo09}
%    {\bodytext{Set up the technical infrastructure, defined the core competences and created a brand together with Peter Duijnstee (the proprietor of Commuun).  Then collaborated on several web applications (all Rails apps) within the context of his company.}}
%}
%
%\headedsection
%  {\href{http://www.eur.nl}{Erasmus University Rotterdam}}
%  {\textsc{Rotterdam, The Netherlands}} {%
%  \headedsubsection
%    {Guest Lecturer}
%    {Jul \apo06 -- Jul \apo09}
%    {\bodytext{Conducted a guest lecture on the phenomenon of open source, as part of the first year curriculum of \emph{Computer Science \& Economics}.}}
%}
%
%%\headedsection
%%  {LIP Automatisering}
%%  {\textsc{Breda, The Netherlands}} {%
%%  \headedsubsection
%%    {Software Auditor}
%%    {Sep \apo06}
%%    {\bodytext{Audited their flag ship product \emph{\acr{LIP} Suite}:\ an %\acr{ERP} solution for construction companies.}}
%%}
%
%\headedsection
%  {\href{http://www.thehealthagency.com}{The Health Agency}}
%  {\textsc{Delft \& Rotterdam, The Netherlands}} {%
%
%  \headedsubsection
%    {Software Engineer}
%    {Jun \apo05 -- Feb \apo06}
%    {\bodytext{Worked on their CMS (written in Python and uses Postgre\acr{SQL}, \acr{XML}/\acr{XSLT} and Twisted).}}
%
%  \headedsubsection
%    {Software Auditor}
%    {Dec \apo06}
%    {\bodytext{Assessed their Python/Zope/\acr{Z}o\acr{DB}-based web framework re-engineering project.}}
%}
%
%\begin{center}
%  \emph{\small Please refer to my \href{http://www.linkedin.com/in/ciesbreijs}{Linked-in profile} for a more complete list of work experiences along with recommendations.}
%\end{center}


%\spacedhrule{1.6em}{-0.4em}
%
%\roottitle{Interests}
%
%\inlineheadsection
%  {Non-exhaustive and in alphabetical order:}
%  {art, Buddhism, cryptography, Go (board game), history, music, open source, philosophy, software engineering (methodologies), travel, typography (e.g.\ graphic design, \LaTeX), \acr{UI}/\acr{UX}-design and vegetarian/vegan cooking.}
%

\end{document}

＀
